\section{Introduction}

The charged K mesons, or kaons are light mesons containing strange (anti-strange) and up (anti-up) valence quarks.  Several properties are summarized in the table below, one important property (for identification) is the lifetime.  The charged kaons have a long enough lifetime to be directly detected in CLAS (for the observed momenta that is detected).

\begin{center}
  \begin{tabular}{| c | c |}
    \hline
    Parameter & Value ($K^{+}/K^{-}$) \\ \hline \hline
    mass & $493.667 \pm 0.002 \; MeV/c^{2}$ \\ \hline
    lifetime & $(1.238 \pm 0.002) \times 10^{-8} \; s$ \\ \hline
    $J^{PC}$ & $0^{-}$ \\ \hline
    quark composition & ($u\bar{s} / s\bar{u}$) \\ \hline
  \end{tabular}
\end{center}

The biggest challenge in identifying charged kaons is separating them from charged pions at higher momenta.  This is evident from figure \ref{fig:kp_bvp}. 
