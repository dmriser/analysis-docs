% --------------------------------------------------------- 
%    The entire document is in this file.  I got lazy.
%
%    1. Introduction
%
% --------------------------------------------------------- 

\section{Introduction}
This document summarizes work done in the 2017 calendar year at Jefferson Lab in Newport News, VA under the supervision of Dr. Kyungseon Joo.  The projects described in this summary are divided into 3 categories  (professional development, CLAS collaboration work, and data analysis work).  First, I will describe professional development activities including presentations and summer schools.  Then, two projects which directly support the efforts of CLAS12 collaboration will be described.  Finally, I will describe our progress in data analysis in terms of basics, large projects, and small projects.  

% --------------------------------------------------------- 
%
%    2. Professional Development 
%
% --------------------------------------------------------- 
\section{Professional Development}
Participation in educational events such as conferences and summer schools, as well as giving presentations and writing publications is a vital part of the graduate student (and scientist) experience.  For the purposes of this summary, I have grouped these activities together under the heading of professional development.  In the past year, I gave 3 presentations of our work, attended 1 conference and 2 summer schools.  
\subsection{Presentations}
In March, I shared our groups work on inclusive cross section extraction from E1-F data.  This presentation was given at the CLAS collaboration meeting during the Deep Processes Working Group (DPWG) section.  During the presentation, preliminary inclusive cross section results were shared in comparison with well known models.  The presentation stressed our methodology for doing acceptance, radiative corrections, and elastic subtraction.  Finally, our future plans to extract the SIDIS cross section were described in the context of the material presented.\\
At the DPWG section of the June collaboration meeting, I presented our recent work on beam spin asymmetries from inclusive kaon events collected during the E1-F run.  The presentation stressed our methodology for identifying kaons, and defended our restriction of the phase space to only low momentum kaons.  Finally, beam spin asymmetry results were shown as a function of 4 kinematic variables for both charged pions and charged kaons.  \\
During the DPWG meeting in October, I presented slides prepared by Dr. Kyungseon Joo which described our exciting recent work with exclusive pi+ neutron events from the E1-F dataset.  The presentation excelled at making connections between variables used in SIDIS and exclusive measurements, as well as demonstrating the qualitatively expected behavior of the beam spin asymmetry as a function of the [cosine of the] polar angle in the pion center of mass frame.  

\subsection{Conferences \& Schools}
\easyFigure{image/photos/MLHEP2017.pdf}{Group photograph from MLHEP 2017.}
Early in the calendar year, several students (myself included) from our research group attended a short workshop at Jefferson Lab entitled 3-D Nucleon Tomography.  Distinguished physicists and computer scientists presented on all topics related to the extraction of 3-dimensional representations of quarks inside of the nucleons.  The workshop focused heavily on Generalized Parton Distributions (GPDs), and offered a complete picture of the status of the field.  \\
During the latter half of June, I attended the first annual Transverse Momentum Distribution (TMD) summer school.  This rigorous summer school was organized by the TMD Collaboration and hosted by Temple University in Philadelphia.  The TMD summer school offered a broad survey of most state-of-the-art aspects of TMD physics.  Fundamental theoretical topics such as the formulation of TMDs in QFT, the evolution of TMDs, and the phenomenology of TMDs were described in great detail by leaders in the field.  The current status of experimental extraction of TMDs was also discussed in detail.  Finally, exciting new lattice calculations of quasi-PDFs were described.  This summer school offered great educational value for students studying TMD physics, as well as an opportunity to meet with some of the leaders in the field.  \\
In July, I attended the 3rd annual Machine Learning in High Energy Physics (MLHEP) summer school organized by Yandex and Imperial College London.  This year, the school was held at the University of Reading in Reading UK.  The 1-week school covered a large variety of concepts in machine learning, such as supervised learning techniques,  unsupervised learning techniques, data processing and transformation, methods to prevent over-training, and optimization problems.  The MLHEP school also invited guest speakers from CERNs LHC-b experiment, as well as ALICE.  These speakers (as well as representatives from Yandex) described the many applications of machine learning at the LHC (trigger systems, particle identification, event selection).  The school format was designed with the student in mind, and offered morning lectures followed by afternoon hands on tutorial sections.  Finally, a physics data analysis challenge was hosted on the data science website Kaggle for participants of the school to apply the techniques taught in class.  Overall this school supplemented my knowledge of statistics, and added increasingly popular new techniques to my data analysis toolbox.    

% --------------------------------------------------------- 
%
%    3. CLAS Projects
%
% --------------------------------------------------------- 
\section{CLAS Collaboration Projects}
In the following section I describe my involvement in projects which directly support the goals of the CLAS collaboration.  My major contributions to CLAS during 2017 have been with the HTCC detector group, and the EVA software project.  

\subsection{HTCC}
During 2016, I spent considerable amounts of time with the HTCC detector group lead by Youri Sharabian.  In the 2017 calendar year, I continued to work with the HTCC group as it was prepared for data taking.  In the beginning of the year, the detector was moved from the TED-F high bay to experimental Hall-B.  The detector was craned into a custom cart and slowly towed to the truck ramp of the hall.  After entering the hall, the detector was lifted with the hall crane into place in the beam line.  Finally, the detector was connected to the electronics in the hall.  Following the movement and installation of the detector, it performed well during the February KPP run.  After the run-period the detector was placed back on the hall floor and re-cabled.  Finally, the detector was re-installed for the engineering run at the end of 2017.  During this process Nick Markov, Will Phelps and I, handled the de-cabling, packing, and re-cabling of all electronic systems for HTCC.  

\subsection{EVA}
One of the scientific cornerstones of CLAS12 operation is the measurement (extraction from observable structure functions) of Transverse Momentum Dependent Parton Distribution Functions (TMD PDFs).  EVA is a software framework for extracting and validating (EVA) TMDs from observables such as asymmetries and cross sections.  The benefit to CLAS of such a project is three-fold.  First, theoretical predictions can be made for observables based on different TMD models.  Second, the software framework can fit parameters of different models to CLAS12 experimental data.  Finally, the optimally tuned software can be used to generate realistic SIDIS events for CLAS12 (and other) simulations.  \\
%My initial involvement in the EVA project was to design and train a neural network regression model that takes kinematic variables as inputs, and outputs the value of a structure function.  The calculation of observable structure functions formally involves a convolution between TMD PDF’s and Fragmentation Function PDF’s (FF), which computationally can be costly because it involves a numerical integration at every phase space point.  We decided to train and query a neural network (NN) to produce the model predictions for cases when full computation became slow.  I used the Keras python library to create a simple neural network with 16 densely connected neurons (4 x 4).  The network was trained by pulling samples randomly from the model and minimizing the mean squared error between the network prediction and the true structure function value at the random point.  The structure function $F_{UU}$ was used for testing and a mean absolute percent error (MAPE) of 10\% was achieved.  These efforts were suspended when we observed that the Gaussian model predictions were faster than the neural network predictions (because the convolution is done analytically). \\
%I then became involved in tuning the parameters of the Gaussian approximation for the software framework.  I was responsible for converting experimental data into the correct format to be read by our framework, as well as making the modifications in python needed to fit the Boer-Mulders and Pretzelosity functions from data.  Working together with Nobuo Sato and Kemal Tezgin, we parametrized and fit the Boer-Mulders and Pretzelosity TMDs using HERMES and COMPASS data.  Finally, we began constructing a simple event generator for CLAS12 use, and it is currently under construction.   

% --------------------------------------------------------- 
%
%    4. Data Analysis Projects
%
% --------------------------------------------------------- 
\section{Data Analysis Projects}
Analysis of E1-F data will be the core content of my PhD thesis.  For this reason, the majority of my time this year was used for data analysis.  For the purpose of this report, my projects have been categorized as basics, large projects, or smaller exploratory projects.  Basic work that I completed this year includes improving hadron identification, a Faraday cup study to determine the beam current during the experiment, and an extension of my code to accommodate E1-6 data.  My efforts to extract the inclusive cross section, calculate kaon beam spin asymmetries, and calculate the beam spin asymmetry for exclusive $\pi^+$ neutron scattering represent the largest projects that I worked on this calendar year.  Finally, small studies of Bethe-Heitler events and lambda (1520) events were made.  These categories are described in more detail in the following 3 sub sections.  

\subsection{Basics}

\subsubsection{Hadron ID}
\easyFigure{image/analysis/identifyKPlus.pdf}{The $\beta(p)$ resolution shown for kaons.}
Hadron identification is done using time-of-flight information in CLAS.  During the past year, I improved the methodology that I use to identify charged pions and kaons as well as protons.  This effort started when we began studying charged kaon asymmetries.  The new method of identification uses the tracks beta value and calculates the likelihood for each possible hadron (as a function of momentum), then classifies the track by maximizing the likelihood ratio.  Finally, the significance can be used to discard tracks which have very unlikely $\beta$ values.  

\subsubsection{Faraday Cup}
\easyFigure{image/analysis/faradayCharge.png}{Faraday cup accumulation as a function of scalar entry number for one run.}
\easyFigure{image/analysis/faradayEntry.png}{Number of events recorded by the DAQ as a function of scalar entry number for one run.}

In order to normalize the observed yield of events to produce cross sections, the number of electrons coming into the experiment needs to be known.  This information is provided by the Faraday Cup, where charge is deposited downstream of the target during runtime.  The amount of charge deposited per scalar reading is recorded in the TRGS bank of the raw data files as FCUPG2.  The total charge deposited during the run is simply the difference between the last reading and the first reading of the run.  However, there are times when beam is recorded and events are not, and these types of situations introduce potential systematic errors into the calculation of total charge.  For this reason, we use a more robust method to calculate the total charge accumulated in a file or run.  First, we record the charge accumulated during subsequent scalar readings.  Then, we record the number of inclusive events observed in the same window.  We then discard from the event sample any events which occur during time windows where no charge was recorded.  We also discard any charge that was recorded during periods of time where no events were recorded.  Finally, the total charge is calculated by taking the difference between every scalar entry and summing it.  The ratio of charge to number of entries is computed for every file, and a consistency cut is used to exclude files taken under non-optimal run conditions.  Using the faraday cup information provided by this study, we calculated the inclusive cross section (discussed below under large projects). 

\subsubsection{E1-6 Analysis}
During the last 2 months of 2017, I started collaborating with Stefan Diehl to adapt my analysis framework to analyze E1-6 data.  To date, we have modified the reader to accept either ntuple10 or ntuple22 formatted root files.  We are currently working on calibration and validation of electron and particle identification for E1-6.   

\subsection{Large Projects}
\subsubsection{Inclusive Cross Section}

\easyFigure{image/analysis/inclusiveAcceptance.png}{Inclusive acceptance corrections for one bin of $Q^2$ as a function of W.}
\easyFigure{image/analysis/inclusiveBackground.png}{Elastic event subtraction corrections for one bin of $Q^2$ as a function of W.}
\easyFigure{image/analysis/inclusiveBins.png}{The binning used for inclusive cross section extraction is shown here.}
\easyFigure{image/analysis/inclusiveDataSim.png}{Inclusive data and monte carlo events shown together for one bin of $Q^2$ as a function of W.}
\easyFigure{image/analysis/inclusivePhaseSpace.png}{Distribution of $\theta_{lab}$ vs. $p$ for data, simulated inelastic events, and simulated elastic events with radiative Bethe-Heitler events included.  The red line indicates the minimum momentum cut which is applied as a result of putting a cut on $y_{max} = 0.7$.}
\easyFigure{image/analysis/inclusiveRadCorr.png}{Inclusive radiative corrections for one bin of $Q^2$ as a function of W.}
\easyFigure{image/analysis/inclusiveRatio.png}{Inclusive cross section ratio to Keppel model for one bin of $Q^2$ as a function of W.}

My thesis work will provide absolute cross section results for charged pion SIDIS.  In order to build trust in the calculation of the normalization constant for the runs, we calculated the inclusive cross section in the resonance region of W from E1-F data using the same files and same electron identification.  The result agrees to within 5\% of the model by Cynthia Keppel (and other Hall A collaborators).  In order to achieve this we carefully selected inclusive events and performed numerous corrections based on Monte Carlo methods.  \\

Events were selected by identifying electrons and applying the restriction that the momentum transfer squared be larger than 1 (common cut used in JLAB kinematics to get closer to the region where pQCD is expected to be correct).  To avoid radiative elastic events, we applied the restriction that y < 0.7.  This was motivated by inspecting the distribution of events produced by the event generators (\texttt{keppel\_rad} and \texttt{keppel\_norad}) with and without radiative effects.  The phase space was then binned with 10 bins in Q2 from 1.5 - 4.5 and 40 bins in W from 1.08 to 2.1.  \\

Using several Monte Carlo event generators we estimated several corrections to the observed yield.  First, we remove elastic events which radiate and end up in the higher range of W.  This is accomplished by running an elastic event generator with radiative effects and an inclusive event generator with radiative effects for the same beam time.  These simulated events are then compared in every bin to find out the fraction of events which come from inclusive.  This factor is applied as a multiplicative correction in every bin.  Second, we correct for the detector acceptance by simulating events in the detector using radiated inclusive events produced by \texttt{keppel\_rad}.  These events are used to construct the ratio of reconstructed to generated events in every bin.  The resulting correction is applied in every bin.  Additionally, we apply radiative corrections from inclusive events which radiate photons before or after the collision in the target.  This is accomplished by calculating the ratio of radiated to un-radiated events in every kinematic bin.  This ratio is then used as a correction factor.  The magnitude of the acceptance correction was observed to be largest, then radiative corrections, and finally elastic event subtraction corrections.  \\
This project was completed in 2017 and provides a solid foundation for cross section extractions from E1-F.  Additionally, it built our trust in the electron identification used.

\subsubsection{Kaon BSA Measurements}
\easyFigure{image/analysis/kaonBSA.png}{Summary of charged pion and kaon beam spin asymmetries extracted from E1-F.}
Several experiments in CLAS12 will measure kaons.  For this reason, we decided to try and measure charged kaon BSAs from E1-F.  Because no acceptance correction is needed we anticipate these results can be easily verified when CLAS12 data analysis is underway.  In our studies, we extracted the integrated or projected asymmetries where we study the dependence of the BSA as a function of one kinematic variable at a time, and include all observed values of the others.  \\
Kaon SIDIS events were selected by first identifying electrons.  After we selected electrons, the event kinematics were calculated and events were required to be in the DIS region ($Q^2 > 1.0, W > 2.0$).  Then charged kaons were identified and events were saved.  Finally, we apply a maximum momentum cut for positively charged kaons.  Above a momentum of 1.75 GeV it is very difficult to separate kaons from pions.  Additionally, we observe that the higher momentum phase space is dominated by lower missing mass resonances such as lambda (1115), sigma (1192), and lambda (1520).  Thus this momentum cut is quantitatively similar to the missing mass cuts used commonly in SIDIS analyses.  Finally, simple binning was selected in $x$, $Q^2$, $z$, and $P_{T}$ and the BSA was calculated for charged pions and kaons.    

\subsubsection{Exclusive $\pi^+$ Neutron BSA Measurements}
\easyFigure{image/analysis/exclusivePiNBSA.pdf}{BSA results for $ep \rightarrow e\pi^+ X$ events from E1-F, shown here as a function of $\cos\theta_h$.}
\easyFigure{image/analysis/exclusivePiNMissingMass.pdf}{Missing mass spectrum for $ep \rightarrow e\pi^+N$ events.}
\easyFigure{image/analysis/exclusivePiNPhiBootstraps.pdf}{Bootstrap fits to the $\phi_h$ distributions.}
\easyFigure{image/analysis/exclusivePiNThetaPhi.pdf}{Angular distribution of events for $ep \rightarrow e\pi^+N$.  The backward angle is excluded due to an acceptance hole at central values of $\phi_h$. }

\easyFigure{image/analysis/evaBohrMulders.png}{Preliminary fits to HERMES asymmetry data for Boer-Mulders TMD, using EVA.}
The last large project that will be discussed here is the beam spin asymmetry measurement for the exclusive pi+ neutron reaction.  This work was suggested by Harut Avakian based on a similar study that he performed using data from the E1-6 run group.  In this study, we calculated BSAs in the SIDIS variable z, as well as in exclusive channel variables $cos\theta_h$.  \\

Exclusive events were selected by first identifying electrons and positive pions (significance > 0.05).  Then a missing mass cut was placed on the remaining final state X between 0.89 and 1.01.  Events were binned in cosine theta in 12 bins of variable size.  The bins in the backward direction were made slightly larger to accommodate the lack of statistics.  The BSA was calculated in 12 bins of phi for each bin of cosine theta, and fit with a sine function using chi-square minimization.  The error on fit parameters was estimated using the Hessian matrix approach as well as using the bootstrap replica method.  Both methods produced similar errors for the extracted BSA in each kinematic bin.  Systematic errors are studied by varying cuts, selecting random subsets of the data and performing the analysis many times, and doing random helicity studies.  \\

Finally, the result is presented as a function of cosine-theta and will be compared to the results obtained by H. Avakian using E1-6 data. 

\subsection{Small \& Exploratory Projects}
\subsubsection{Bethe-Heitler Events}
During the study of the inclusive cross section we explored Bethe-Heitler events by identifying electrons and protons to try and understand the nature of events which may be radiating into our sample.  Radiative elastic events should still have zero missing mass, and the angle between final state electron and proton should be 180 degrees in the lab frame.  We put cuts on both of these variables to identify elastic and Bethe-Heitler events.  The events which fall outside of the normal W range near the proton mass are then picked as Bethe-Heitler events.  We studied the distribution of y for these events, as well as the distribution of the angles between the missing photon and the initial and final state electrons.  We observed that the BH process is dominated by pre-collisional radiation, a conclusion drawn from the fact that the spectrum of the angle between the beam line and the photon peaks sharply at zero.  Ultimately, by studying events with theta ex > 10 degrees we decided to exclude the region of W > 2.1 from our inclusive cross section study, because it would be too difficult to control the background with the tools on hand.    

\subsubsection{$\Lambda$ (1520) Events}
\easyFigure{image/analysis/lambdaResonance.pdf}{Interesting $\Lambda$ (1520) events recorded during the E1-F run.}
As a test of kaon identification, we decided to investigate some positive kaon resonance channels.  Upon inspection of the missing mass spectrum of $e p \rightarrow e K^{+} X$ we noticed a prominent resonance at 1520 MeV which was later confirmed by a group of experts to be the lambda 1520 resonance.  Since the resonance has not been studied extensively, we decided to isolate it and study the kinematics, as well as do a resonance width and position analysis.  The results agree for both the case when epk are detected, as well as the fully exclusive $e p \rightarrow e p K^{+} K^{-}$.  The resonance width we measure with both methods agrees with PDG estimates, and is 25 MeV less than the width previously quoted with CLAS measurements (Reference 1).

% --------------------------------------------------------- 
%
%    5. Review/Summary 
%
% --------------------------------------------------------- 
\section{Summary}
In this document, I have given some introductory details about all of my project involvement during the 2017 calendar year.  Three collaboration talks were presented on various subjects in electron scattering experiments.  I also attended two summer schools, one on TMD physics and the other focused on the use of machine learning in nuclear and particle physics.  I engaged heavily in several projects that directly benefit CLAS, such as EVA and HTCC detector work.  Additionally, I worked toward my thesis by improving our hadron identification, completing the Faraday cup study and inclusive cross section, and calculating several asymmetries.  

% --------------------------------------------------------- 
%
%    6. Acknowledgements
%
% --------------------------------------------------------- 
\section{Acknowledgements}
I want to acknowledge Dr. Kyungseon Joo for his support in these research projects.  I also wish to thank Youri Sharabian, Nick Markov, Will Phelps, Nobuo Sato, Kemal Tezgin, Andrey Kim, Stefan Diehl, and Brandon Clary for their valuable help.  

\easyFigure{image/photos/htcc_install_1.jpg}{Installation of HTCC.}
\easyFigure{image/photos/htcc_install_2.jpg}{Installation of HTCC.}
\easyFigure{image/photos/htcc_install_3.jpg}{Installation of HTCC.}
\easyFigure{image/photos/htcc_install_4.jpg}{Installation of HTCC.}
