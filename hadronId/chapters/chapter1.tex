\section{Introduction}

% ------------------------------------------
%     introduction body 
% ------------------------------------------

Particle identification is the process of combining detector level information, as well as higher level information from the results of reconstruction (charge, momentum, and time of flight) to catagorize each track as a known particle.  These particles can then be corrected and used in physics analyses. \\

In this note, the authors describe their methodology used to identify positively charged tracks as one of three common species (pion, kaon, proton).  Cuts are applied to remove tracks in geometrically poorly understood regions of the detector, followed by a maximum likelihood ratio based identifiation using time-of-flight information.  An additional cut is used for our analyses which restricts the vertex of the positive track to be close to that of the electron.  Such a cut should be removed to study processes with detached vertex positions (arising from the decay of other produced hadrons). \\

The methodology described in this note is based on the discussion provided by (reference the BESIII paper on likelihood based particle identification).  


